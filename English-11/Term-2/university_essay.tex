%!TEX TS-program = xelatex
    %!TEX encoding = UTF-8 Unicode
%    \documentclass[10pt, letterpaper]{article}
\documentclass[letterpaper,10pt,oneside,headsepline]{scrreprt}
\usepackage{fontspec} 
\usepackage{placeins}
\usepackage{multibbl}
\usepackage{graphicx}
\usepackage{txfonts}
\usepackage{url}
\usepackage{xltxtra}
\usepackage{geometry} 
\usepackage{pdfpages}
\usepackage{amssymb}
\usepackage{makeidx}
\usepackage{paralist}
\usepackage{comment}
    \geometry{letterpaper, textwidth=5.5in, textheight=8.5in, marginparsep=7pt, marginparwidth=.6in}
    %\setlength\parindent{0in}
    \defaultfontfeatures{Mapping=tex-text}
    \setmainfont{Arvo}
    \setsansfont [Ligatures={Common}, BoldFont={Arvo-Bold}, ItalicFont={Arvo-Italic}]{Arvo} 
\usepackage[ngerman,english]{babel}
\usepackage{textcomp}
\usepackage{scrpage2}
\clubpenalty=6000
\widowpenalty=6000
\author{Avery Laird}
\title{English 11: University Essay}
\date{\today}
\ohead{English 11}
\chead{University Essay}
\pagestyle{scrheadings}
\setcounter{secnumdepth}{-1}
\makeindex
\begin{document}

Near the campsite for Elfin Lakes in Garibaldi Park, there is a beautiful green expanse sloping downward; it finds elegance through simplicity, and demands appreciation. The panorama extends for miles, allowing the viewer a breathtaking glimpse of distant glaciers. Here, the unobstructed back country stretches all the way up the west coast -- here, there is true wilderness. It was my second day in this place, and as night fell heavy there seemed a charge to the air that was not wholly figurative -- late into that night, the largest electrical storm in recent history struck the Lower Mainland. Directly overhead. In the driving thunder and rain, a small two by two metre tent was my only protection, and I could see the simultaneous flash and clap of thunder silhouetted against the thin walls. There was nowhere to go -- I spent the night trapped, helpless against the fury of the storm. Eventually, as the elements calmed and day broke, I tentatively left the confines of my tent. Once again, I looked out upon the magnificent landscape, only to find that I perceived a new depth to this wonderfully barren land; my fear from the long night had melted with the sunrise, and I felt a common bond between myself and the rugged terrain. Our shared experiences, forged in a trial by fire of sorts, showed me there is often more to something than the immediately obvious.            

\end{document}
